\documentclass[12pt]{article}
\usepackage{tikz}
\usepackage{verbatim}
\usepackage{listings}
   
\lstset{ %
%language=C++,                % the language of the code
basicstyle=\footnotesize,       % the size of the fonts that are used for the code
keywordstyle=\color{black}\bfseries,
%numbers=left,                   % where to put the line-numbers
%numberstyle=\footnotesize,      % the size of the fonts that are used for the line-numbers
%stepnumber=2,                   % the step between two line-numbers. If it's 1, each line 
%                                % will be numbered
%numbersep=5pt,                  % how far the line-numbers are from the code
backgroundcolor=\color{white},  % choose the background color. You must add \usepackage{color}
%showspaces=false,               % show spaces adding particular underscores
showstringspaces=false,         % underline spaces within strings
showtabs=false,                 % show tabs within strings adding particular underscores
%frame=single,                   % adds a frame around the code
%tabsize=2,                      % sets default tabsize to 2 spaces
%captionpos=b,                   % sets the caption-position to bottom
%breaklines=true,                % sets automatic line breaking
%breakatwhitespace=false,        % sets if automatic breaks should only happen at whitespace
title=\lstname,                 % show the filename of files included with \lstinputlisting;
% also try caption instead of title
%escapeinside={\%*}{*)},         % if you want to add a comment within your code
%morekeywords={*,...}            % if you want to add more keywords to the set
}

\usetikzlibrary{arrows}
\usetikzlibrary{shapes.multipart}
\usetikzlibrary{%
   decorations.fractals%
  ,decorations.pathmorphing%
  ,shadows%
}

\usepackage{hyperref}
\hypersetup{
bookmarks=false,         % show bookmarks bar?
unicode=true,          % non-Latin characters in Acrobat's bookmarks
pdftoolbar=true,        % show Acrobat's toolbar?
pdfmenubar=true,        % show Acrobat's menu?
pdffitwindow=true,     % window fit to page when opened
pdfstartview={FitH},    % fits the width of the page to the window
pdftitle={Swat guide},    % title
pdfauthor={Marcelo Zimbres Silva},     % author
pdfsubject={Spherical Wavelets},   % subject of the document
pdfcreator={Marcelo Zimbres Silva},   % creator of the document
pdfproducer={Marcelo Zimbres Silva}, % producer of the document
pdfkeywords={Wavelets} {Cosmic Rays}, % list of keywords
pdfnewwindow=true,      % links in new window
colorlinks=true,        % false: boxed links; true: colored links
linkcolor=black,          % color of internal links
citecolor=black,        % color of links to bibliography
filecolor=black,      % color of file links
urlcolor=black           % color of external links
}

% My definitions
%______________________________________________________________________

%\colorlet{swat}{yellow!50!black!50}
\colorlet{swat}{brown!50!black!50} % box color
\colorlet{textcolor}{brown!40!black} % text in box color
\colorlet{swatred}{red!50!black}

\tikzstyle{one}=[shade,top color=swat,fill=swat,rounded corners, textcolor,thick]
\tikzstyle{onel}=[rotate=-10,shade,top color=swat,fill=swat,rounded corners, textcolor,thick]
\tikzstyle{oner}=[rotate=10,shade,top color=swat,fill=swat,rounded corners, textcolor,thick]
\tikzstyle{inherit}=[open triangle 45-,thick,textcolor]
\tikzstyle{depend}=[draw,->,thick,textcolor,dashed]
\tikzstyle{code}=[draw,shade,rounded corners,text width=9cm]


\tikzstyle{two}=[anchor=north,shade,top color=swat,rectangle split, 
                     rectangle split parts=2, draw, text width=2.0cm,
		     fill=swat,rounded corners,textcolor,thick]

\tikzstyle{twonowidth}=[anchor=north,shade,top color=swat,rectangle split, 
                     rectangle split parts=2, draw, fill=swat,rounded corners,
		     textcolor,thick]


\tikzstyle{twored}=[anchor=north,shade,top color=swatred,rectangle split,
                        rectangle split parts=2,draw,text width=2.0cm,fill=swatred]

\tikzstyle{three}=[anchor=north,shade,top color=swat,rectangle split, rectangle split parts=3, draw, text width=3.0cm,fill=swat,
rounded corners,textcolor,thick]

\def\one{\node[style=one]}
\def\onel{\node[style=onel]}
\def\oner{\node[style=oner]}
\def\two{\node[style=two]}
\def\twonowidth{\node[style=twonowidth]}
\def\code{\node[style=code]}

\def\inherit{\draw[style=inherit]}
\def\depend{\draw[style=depend]}

\def\twored{\node[style=twored]}
\def\three{\node[style=three]}

%______________________________________________________________________
\begin{document}

  \parindent0pt
  \null
  \colorlet{mintgreen}{green!50!black!50}

  \thispagestyle{empty}
  \vskip3cm
  \vfill
  \hfil
  \begin{tikzpicture}[overlay]
    \coordinate (front) at (0,0);
    \coordinate (horizon) at (0,.28\paperheight);
    \coordinate (bottom) at (0,-.6\paperheight);
    \coordinate (sky) at (0,.62\paperheight);
    \coordinate (left) at (-.51\paperwidth,0);
    \coordinate (right) at (.51\paperwidth,0);

    \shade [bottom color=blue!30!black!10,top color=blue!30!black!50] ([yshift=-5mm]horizon -|  left) rectangle (sky -| right);

    %\shade [bottom color=black!70!green!25,top color=black!70!green!10] (front -| left) -- (horizon -| left)
    %   decorate [decoration=random steps] { -- (horizon -| right) } -- (front -| right) -- cycle;

    \shade [top color=black!70!green!25,bottom color=black!25] (front -| left) -- (horizon -| left)
      decorate [decoration=random steps] { -- (horizon -| right) } -- (front -| right) -- cycle;

    %\shade [top color=black!70!green!25,bottom color=black!25] ([yshift=-0mm]front -| left) rectangle ([yshift=1pt]front -| right);

    \fill [black!25] (bottom -| left) rectangle ([yshift=3mm]front -| right);

    \def\nodeshadowed[#1]#2;{\node[scale=2,above,#1]{#2};\node[scale=2,
        above,#1,yscale=-1,scope fading=south,opacity=0.4]{#2};}

    \node[at={(-0,6  )}] {\Huge \textsc{\bf \textcolor{red!50!black!90}{SWAT}}}; 
    \node[at={(-0,4  )}] {\Huge {\bf\color{gray!70!black} The \textcolor{red!50!black!90}{S}pherical \textcolor{red!50!black!90}{W}avelet \textcolor{red!50!black!90}{A}nalysis \textcolor{red!50!black!90}{T}ool}};

    \one (TObject) at (-8,15) {\Large \bf\color{gray!70!black} TObject};
    \one (TArrayD) at (-3,15) {\Large \bf\color{gray!70!black}TArrayD};
    \one (TDKernel) at (2,15) {\Large \bf\color{gray!70!black}TDKernel};
    \one (TCoeffInfo) at (7,15) {\Large \bf\color{gray!70!black}TCoeffInfo};

    \one (TNamed) at (-8,13) {\Large \bf\color{gray!70!black}TNamed};
    \one (TVMap) at (-3,13) {\Large \bf\color{gray!70!black}TVMap};
    \one (TEulerAngle) at (2,13) {\Large \bf\color{gray!70!black}TEulerAngle};
    \one (TAngDistance) at (7,13) {\Large \bf\color{gray!70!black}TAngDistance};

    \one (TSkyMap) at (-8,11) {\Large \bf\color{gray!70!black}TSkyMap};
    \one (TWignerd) at (-3,11) {\Large\bf\color{gray!70!black} TWignerd};
    \one (TWavMap) at (2,11) {\Large \bf\color{gray!70!black}TWavMap};
    \one (TAlm) at (7,11) {\Large \bf\color{gray!70!black}TAlm};

    \one (TSphHarmB) at (-8,9) {\Large \bf\color{gray!70!black}TSphHarmB};
    \one (TSwatB) at (-3,9) {\Large \bf\color{gray!70!black}TSwatB};
    \one (TSwatF) at (2,9) {\Large \bf\color{gray!70!black}TSwatF};
    \one (TSphHarmF) at (7,9) {\Large \bf\color{gray!70!black}TSphHarmF};
   
   %_______________________________________________________
   \draw[draw,-open triangle 45,thick,color=gray] (node cs:name=TNamed) -- (node cs:name=TObject);
   \draw[open triangle 45-,thick,gray] (node cs:name=TDKernel,anchor=south) |- +(0,-1)  node (y) {};
   \draw[open triangle 45-,thick,gray] (node cs:name=TCoeffInfo,anchor=south) |- +(0,-1)-- +(-10.0,-1);
   \draw[open triangle 45-,thick,gray] (node cs:name=TArrayD,anchor=south) -- (node cs:name=TVMap);
   \draw[draw,-open triangle 45,thick,gray] (node cs:name=TVMap) -- (node cs:name=TNamed);
   \draw[draw,-open triangle 45,thick,gray] (node cs:name=TSkyMap) -- (node cs:name=TVMap);
   \draw[draw,-open triangle 45,thick,gray] (node cs:name=TWavMap) -- (node cs:name=TVMap);
   \draw[draw,->,thick,gray,dashed] (node cs:name=TWavMap) -- (node cs:name=TSwatB) node[midway,sloped,above]           {};
   \draw[draw,->,thick,gray,dashed] (node cs:name=TWavMap) -- (node cs:name=TAlm) node[midway,sloped,above]             {};
   \draw[draw,->,thick,gray,dashed] (node cs:name=TSkyMap) -- (node cs:name=TSphHarmB) node[midway,sloped,above]        {};
   \draw[draw,->,thick,gray,dashed] (node cs:name=TSwatB) -- (node cs:name=TWignerd) node[midway,sloped,above]          {};
   \draw[draw,->,thick,gray,dashed] (node cs:name=TSphHarmB) -- (node cs:name=TWignerd) node[midway,sloped,above]       {};
   \draw[draw,->,thick,gray,dashed] (node cs:name=TWavMap) -- (node cs:name=TEulerAngle) node[midway,sloped,above]      {};
   \draw[draw,->,thick,gray,dashed] (node cs:name=TAngDistance) -- (node cs:name=TEulerAngle) node[midway,sloped,above] {};
   \draw[draw,->,thick,gray,dashed] (node cs:name=TAlm) -- (node cs:name=TSwatF) node[midway,sloped,above]              {};
   \draw[draw,->,thick,gray,dashed] (node cs:name=TAlm) -- (node cs:name=TSphHarmF) node[midway,sloped,above]           {};

   %_______________________________________________________

   \node[at={(0,-3.0)}] {\includegraphics[scale=2.0]{skymap.pdf}};

   \node[at={(0,-10)}] {\Large\bf \color{gray!70!black}Manual for version 1.53 - Marcelo Zimbres};
   %\draw[-,very thick,gray] (-9,-10.5) -- (9,-10.5);

  \end{tikzpicture}
  \vfill


\newpage
\tableofcontents

\section{Introduction} \label{ch::Introduction}

\href{http://www.ifi.unicamp.br/~mzimbres}{SWAT} is a package for analysis of
functions that are defined on the sphere. It has been used extensively to
analyze data collected by the Pierre Auger experiment, CRPropa simulations and
other simulation data. The program has been written from scratch as part of my
my Ph.D, where it evolved from a simple ROOT macro and got bigger and bigger,
since the code may be useful to other people I decided to organize, document it
and make it public. The project main feature is its C++ implementation of the
spherical wavelet transform as presented in \cite{wiaux}.  Some attractive
features of SWAT are:

\begin{list}{\labelitemi}{}
\item Harmonic and Wavelet transform on the sphere.
\item Interface to Healpix code, which is included in the build system (the user
does not have to install it).
\item Easy selection of events hitting sky window. 
\item Calculation of deflection vs. $1/E$ graphs.
\item ROOT and FFTW are the only prerequisites.
\end{list}

The package includes some parts of the Healpix code. There are two reasons why
I decided to include it here, instead of just link against Healpix libraries.
\begin{enumerate}
\item I do not need all Healpix routines and support to the fits format. 
\item I usually need shared libraries to call Healpix code in a ROOT session,
which are not built by Healpix build system since most of its code are C++
templates.
\item Healpix installation used to be messy.
\end{enumerate}

Most of the theoretical details of analysis of functions defined on $S^2$ and
$SO(3)$ were taken from \cite{wiaux} and references therein. 

The code has been tested in \textsc{Linux} and \textsc{MacBook}, but the code
is portable enough to be built on other platforms.  In the following we
describe how to use SWAT. Questions concerning the software can be sent to Marcelo
Zimbres \href{mailto:mzimbres@gmail.com}{mzimbres@gmail.com}
\vspace{0.7cm}
\newline
{\bf \large Acknowledgements}
\vspace{0.7cm}

Most of SWAT were developed with financial support from CAPES.  Additionaly I
would like to acknowledge the Institute of Physics at UNICAMP and the Bergische
Universit\"at Wuppertal for finacial support, Brunel University for granting me a
scholarship to participate in the Cern School of Computing, the Ettore Majorana
Foundation and Center for scientific Culture, the brasilian group of the Pierre
Auger experiment, with special thanks to my supervisor Ernesto Kemp, for
beliving in the project and Rafael Alves Batista for testing the code. The
Pierre Auger Group at university of Wuppertal, with special thanks to Nils
Niertenh\"ofer for helping me with CRPropa.

\subsection{Installation} \label{ch::installation}
As a prerequisite to install SWAT you need ROOT and FFTW installed, the
configure script will fail if they are not installed. SWAT uses autotools to
generate the configure script and the Makefile, so you can expect all the
standard configure options and makefile targets. The lines bellow will install
the libraries on {\color{textcolor}``/usr/local"}
{\bf \color{textcolor}
   \begin{lstlisting}
   $ tar -xvzf swat-1.53.tar.gz
   $ cd swat-1.53
   $ ./configure CXXFLAGS="-O3 -ffast-math"
   $ make
   $ sudo make install
   \end{lstlisting}
}
The flags passed to the configure script are optional, but greatly improve
performance.  To use some of the features of the package, I am assuming you are
able to build shared libraries on your platform. To easy the task of loading
swat libraries the macro load.C, will be installed on
{\color{textcolor}"/usr/local/share/swat"}. To load the libraries in ROOT's C++
interpreter you have to execute it in your ROOT session, or add it to
your code {\color{textcolor}.rootlogon.C} macro.

{\bf \color{textcolor}
\begin{lstlisting}
$ cp /usr/local/share/swat/load.C ~/.rootlogon.C
$ root # The .so's are automatically loaded here.
\end{lstlisting}
}

All SWAT classes are now available in the ROOT session with syntax highlighting
(currently I do not use swat from inside a root session, but it may be usefull
for others). You should also be able to generate documentation in HTML format
with the macro {\color{textcolor}"prefix/share/swat/makehtml.C"}. You have to run
this macro from the directory where you built SWAT. The documentation will be
built in the directory {\color{textcolor}htmldoc}, this is a nice way to get aquainted with 
the code (this is meant for those wanting to develop).

\subsection{Getting ready} \label{ch::ready}

To use Herald file (a file distributed by the Pierre Auger collaboration containing collected data),
you will have to convert the ascii file to a {\bf TTree} and
save it in a .root file, for that you should use the macro
{\color{textcolor}prefix/share/swat/convert\_herald.C} where prefix is usually
{\color{textcolor}/usr/local}.  Copy this macro to the same directory where you have the
herald data file. The macro will read a file with name
{\color{textcolor}"herald.dat"}, so you may have to rename your file or edit the macro. 
For CRPropa simulations you can configure CRPropa to output a .root file
instead of a text file.  The code uses the title of the {\bf TTree} to differentiate
between a Herald and CRPropa file. The title of the TTree for CRPropa must be
``{\color{textcolor}CRPropa 3D events}", as far as I know, this is the default.

\subsection{High quality graphs with pgfplots} \label{ch::pgfplots}
If you have pgfplots installed on your machine, you can use the \LaTeX\,\,
files which are available in the directory {\color{textcolor}pgfplots} in swat
root dyrectory, to generate high quality graphs using some output of swat
analysis.  You may have noted that after running {\color{textcolor}swat\_find}
you get many text files in your working directory. They are input files for the
\LaTeX\,\, files. In the following subsections we show how to use the material.
\subsubsection{Skymaps}
When you run the program {\color{textcolor}swat\_find} or {\color{textcolor}swat\_sim}, they produce a file named
{\it skymap.dat}. This file contains the coordinates and energy of the events that
passed the cut. To generate a graph for an isotropic sky for example, one can
use:
{\bf \color{textcolor}
\begin{lstlisting}
$ tar -xzf skymap.tar.gz
$ cd skymap
$ swatsim -n 1000 -s 1
$ pdflatex --jobname=skymap-f1 skymap.tex
\end{lstlisting}
}
Example skies can be seem here:\\
\includegraphics[scale=1.0]{skymap-sim.pdf} 
\includegraphics[scale=1.0]{skymap.pdf}
The program will also generate the file mult\_cand.tex, this file contains only
the events which hit the tangent plane in the position found by the wavelet
analysis. An example can be seem below on the left side. On the right side we
see a skymap simulated with CRPropa. \\
\includegraphics[scale=1.0]{mult-cand.pdf} 
\includegraphics[scale=1.0]{crpropa.pdf}

Usually the energy of events is represented in skymaps as circles of varying
size, where events with higher energy have larger circles. I do not like such
representations since they can give the false impression that the angular
resolution gets worse as the energy increases, which is not true. Additionally,
these circles pollute the figure. In the representations I am using, colors
are used to differentiate energies which in my opinion is a much clear way of
representing the sky.

\subsubsection{Energy deflection graphs}
In addition to skymap.dat, the files {\color{textcolor}corr\_graphN.dat} will be also generated, 
with N varying from $0$ to $15$ (this number can be passed in the command line). These files contains the energy-deflection graphs,
of the events the hit the tangent plane in the position found by the wavelet
analysis. On figures \ref{deflections}, we see two examples. The graph on the left
originates from a CRPropa simulation. \\
\begin{figure}
\centering
\includegraphics[scale=0.8]{corr_graph.pdf} 
\includegraphics[scale=0.8]{corr_graph-sim.pdf}
\label{deflections}
\end{figure}

%\subsubsection{Number of events versus orientation}
%
%The number of events hitting stripes centered at the same point and
%whose orientation varies from 0 - 180 degrees is also a useful observable
%when interpreting the results. Some of these graphs can be seem below:
%
%\includegraphics[scale=0.8]{multiplicity.pdf}  
%\includegraphics[scale=0.8]{multiplicity-sim.pdf} 
%
%To generate them, use the macro {\color{textcolor}multiplicity.tex} in the pgfplots directory
%and the file {\color{textcolor}multiplicity0.dat} generated by swatfind of swatsim.
%One such file will be generated for each source found.
%Similar to what we have done for the energy deflection graphs.
%

\subsubsection{Wigner-d functions}

Swat implements the calculation of the wigner $d^l_{mn}$ functions via FFT.
You can use the file pgfplots/wignerd.tar.gz to generate some nice graphs.

\begin{figure}
\centering
\includegraphics[scale=1.0]{wignerpolar.pdf}
\includegraphics[scale=1.0]{wigner.pdf} 
\end{figure}
\section{Transforms on the Sphere}
Most SWAT functionality is based on Fourier and wavelet transforms on the sphere, therefore we give
in this section a brief overview of analysis of functions on the sphere. Before
that however, it is important to mention the problem of sampling functions on
the sphere, which is refered to as its pixelization.  Due to the fact
that most of the time we work with function defined on the line (temporal
series, for example) or on the plane (images in general) we do not have to care
much about how to sample. We simply divide it in squares of same area and everything works
just fine. However, when we try the same thing on the sphere we imediately see
that it is impossible to achieve same area pixelization on the sphere due its curvature.

There are two main features we would like to have when dividing a sphere in
pixels. First we would like to have pixels of same area, this is achieved by the
Healpix pixelization \ref{healpix}. Second, we would like to to use FFT to calculate
functions on the sphere instead of slow recursion and that is achieved by ECP
Pixalization, which stands for ``Equidistant cylindrical projection"\ref{healpix}.
Unfortunately theses two features can not be achieved at the same time. In this software we
decided to use ECP due to its simplicity. In ECP, the three Euler angles, which we 
are using to parametrize SO(3), are sampled as follows
$\alpha$, $\beta$ and $\gamma$
\begin{equation}
\alpha_{j_1} = \gamma_{j_1} = \frac{2\pi j}{2B}, \ \ \ \beta_{j_1} = \frac{\pi(2k + 1)}{4B}
\nonumber
\end{equation}
\begin{figure}[ht]
   \centering
      \includegraphics[height=4cm,width=10.78cm,angle=0]{healpixGridRefinement.jpg} \\
      \includegraphics[]{cylindrical-f1.pdf}
      \includegraphics[]{cylindrical-f2.pdf}
      \includegraphics[]{cylindrical-f3.pdf}
   \caption{Above:Healpix pixelization of the sphere. Below: The ECP pixelization.}
   \label{healpix}
\end{figure}

\subsection{Fourier Transform on SO(3)}
A function $f \in L^2(SO(3))$ can be decomposed as follows
\begin{eqnarray}
f(\alpha,\beta,\gamma) = \sum_{l = 0}\sum_{m = -l}^l\sum_{n = -l}^l f^l_{mn}D^l_{mn}(\alpha,\beta,\gamma).
\label{back}
\end{eqnarray}
where
\begin{eqnarray}
D^l_{mn}(\alpha,\beta,\gamma) = e^{-im\alpha}d^l_{mn}(\beta)e^{-in\gamma}.
\end{eqnarray}
is called "Wigner-D functions" and form an irreducible representation on SO(3). The
functions $d^l_{mn}$ are called "small wigner-d functions".

The inverse of \ref{back} on ECP pixelization is given by
\begin{equation}
f^l_{mn} = \frac{1}{(2B)^2}\sum_{j_1 = 0}^{2B - 1}\sum_{j_2 = 0}^{2B - 1}\sum_{k = 0}^{2B - 1}
w_B(k)f(\alpha_{j_1},\beta_{k},\gamma_{j_1})D^{l*}_{mn}(\alpha_{j_1},\beta_{k},\gamma_{j_1})
\end{equation}
where $B$ is the band limit and the quadrature weights are given by
\begin{equation}
w_B(k) = \frac{2}{B}\sin\left(\frac{\pi(2k + 1)}{4B}\right)\sum_{j = 0}^{B - 1}
\frac{1}{2k + 1}\sin\left((2j+1)(2k+1)\frac{\pi}{4B}\right)
\end{equation}
where $0\le k < 2B$. 

\section{Installed programms}

In the following sub-sections we will describe the main programs which 
SWAT installs and which provide an easy interface to the wavelet analysis.

\subsection{swat\_find}

{\it swat\_find} is the main program. It can be used to find sources in Pierre
Auger data and CRPropa simulations. Beyond that, it can be also used to test
whether the events belong to a multiplet, calculating energy-deflections graphs
for the sources found. All information is saved in .root format. A source here
is a term used to mean something with a position on the sky, usually described
by $(\theta,\phi)$ and an orientation. We will use the three Euler angle
$(\alpha,\beta,\gamma)$ to describe the source. 

It uses the following algorithm:
\begin{enumerate}
\item The TTree containing Pierre Auger data or CRPropa simulations
is read from the file passed in the command line with option -f.
\item Events with energy in the range $emin < E < emax$ are selected and used
to fill a Healpix map.  The options -i and -e are used to pass the energy range.
\item The Healpix map generated is transformed to wavelet space. The parameter
$N$, passed by the option -N gives the precision on the ability of the wavelet
to find the angle $\gamma$, which can range from 1 - 128 in our implementation. 
The precision in degrees are calculated with the formula $180/N$ The scale on
which the analysis is performed is passed with option -j. It ranges from 0 - 8.
We found out that the best results are achieved with j = 1.
\item A partial sort is used to find the 15 largest wavelet coefficients.
\item The euler angles found will be used to calculated the tangent plane
equation for each source. The width and length of the plane are passed with
options -w and -l respectively.
\item A loop on the data is made to select all events hitting the tangent
plane. The energy cut is used again.
\item The energy-deflection graphs are calculated and the correlations found
are printed on the screen.
\end{enumerate}
All the information is saved in the file {\color{textcolor}"sources.root"}.  Lets see an example:

{\bf \color{textcolor}
\begin{lstlisting}
$ swatfind -j 1 -N 127 -i 20 -e 40 -w 2 -l 10 -f chain.root
TFile**		chain.root	CRPropa output data file
 TFile*		chain.root	CRPropa output data file
  OBJ: TNtuple	events	CRPropa 3D events
  OBJ: TEventList	list	20 < emin && emax < 40 
  OBJ: THealpixMap	hmap	Healpix sky map
  OBJ: TGraph	g0	C = -0.993, N = 4 
  OBJ: TGraph	g1	C = -0.986, N = 5
  OBJ: TGraph	g2	C = 0.000, N = 0
  OBJ: TGraph	g3	C = -1.000, N = 2
  ...
\end{lstlisting}
}

The N in the TGraph title is the number of events in the graph (or the number of 
events which hit the tangent plane) and is not to be confused with the N passed by command line
option -N. C is the correlation of the energy-deflection graphs.

Now, if you want to see the exact location of the source:
{\bf \color{textcolor}
\begin{lstlisting}
$ root sources.root
root[1] .ls
TFile**		sources.root	
 TFile*		sources.root	
  KEY: THealpixMap	hmap;1	Healpix sky map
  KEY: TEulerAngle	source0;1	
  KEY: TEulerAngle	source1;1	
  KEY: TEulerAngle	source2;1	
root [2] source0->Show(1)
wav(137.812,-27.4219,159.638) = -0.035312
root [3] 
\end{lstlisting}
}

Help description
{\bf \color{textcolor}
\begin{lstlisting}

Searches for multiplets in herald data or CRPropa
simulations. Calculates correlation of events hitting
stripe on the tangent plane and dispalys on the screen.
The results are saved to sources.root file. Two input
file formats are supported, both are TTrees saved in a
.root file. The TTrees can be either the output of
CRPropa or a Herald file converted to TTree (see macro
macros/convert_herald.C in swat source tree).

Usage: swat_find [ -j scale] [-N number] [-i emin] 
       [-e emax] [-w width] [-l length] [-f file.root] 
       [-n nsources] [-t wav_threshold]

Options:

-h:     This menu.
-j:     Wavelet scale. It is a number in the range 
        0 <= j <= 8, defaults to 1.
-N:     Band limit of wavelet, in the range 0 < N <= 128,
        defaults to 1.
-i:     Minimum energy of events, defaults to 20 EeV.
-e:     Maximum energy of events, defaults to 40 EeV.
-w:     Width of tangent plane, defaults to 2 degrees.
-l:     Length of tangent plane, defaults to 10 degrees.
-f:     Root file containing Tree with data, defaults to
        chain.root.
-n:     Number of sources to look for. Default to 15
-t:     Wavelet threshold value.
\end{lstlisting}
}
\subsection{swat\_sim}
{\it swat\_sim} uses Monte Carlo simulations to calculate the probability of a
multiplet happen by chance on a isotropic sky. The distribution of $E$, $\theta$
and $\phi$ must be passed in the command line, it can be generated by swat\_gen The
probability is calculated as follows: 
\begin{itemize}
\item The program simulates n isotropic skies. If you want to add your simulated events
to each sky you can use the option -f and a TTree in CRPropa format will be read
and added in each sky.
\item For each sky simulated the analysis made by swat\_find is used to calculate the
correlation coefficient. 
\item If the correlation is larger than C , passed with option -c and has more than n events, passed 
with option -m, then the multiplet has passed the criteria.
\item The probability will be total number of multiplets passing the criteria divided by 
number of skies simulated.
\end{itemize}

Beyond the probability, the program outputs two additional histograms:
\begin{enumerate}
\item A histogram of the correlation coefficients found, for which the total number
of events is larger than the value passed with -m.
\item A histogram of the number of events in the tangent plane.
\item A histogram of the largest wavelet coefficients found.
\end{enumerate}

Help message

{\bf \color{textcolor}
\begin{lstlisting}

Calculates the probability of a multiplet with minimum
correlation c > c_0 (see -c option), minimum number of
events m > m_0 (see -m option) and where the magnitude of
the wavelet coefficint e C > C_0 (see -C option), happen
by chance using wavelet analysis.  First an isotropic sky
is simulated (the coverage and energy distribution must
be provided) and the wavelet representation of the sky is
calculated, the euler angles of the largest coefficient
is used to calculate the equations of the tangent plane
at the position found (the euler angles). The correlation
c is calculated including all events that hit the tangent
plane, whose size is specified with the options -l and
-w. The probablility will be the number of multiplets
with c > c _0, C > C_0 and m > m_0, divided by the number
of skies simulated. Additionaly, seven other quantities
are calculated:
   
   1 - The histogram of the number of events that hit the
       tangent plane.
   2 - The histogram of the c's found for which the
       number of events is greater than m_0 and C > C_0
       (passed in the command line).
   3 - The histogram of the magnitude of wavelet
       coefficients C_0.
   4 - The histogram of the mean of wavelet coefficients.
   5 - The histogram of the variance of wavelet
       coefficients.
   6 - The histogram of the skewness of wavelet
       coefficients.
   7 - The histogram of the kurtosis of wavelet
       coefficients.
   
If -f option is used, a TTree in the file will be read
and events will be added to the analysis, this is useful
to include a simulated multiplet on the analysis, hiding
it in the isotropic backgroung the test the algorithm.

It is also mandatory to specify:

   - Energy distribution.
   - The theta distribution.
   - The phi distribution.

This is the distribution the background events have to
follow. These distributions are read from a root file.
Use swat_gen and swat_coverage to generate them.

Usage: swat_sim [ -j scale] [-N number] [-n nevents] 
       [-s skies] [-i emin] [-e emax] [-c corr] [-m mevents]
       [-w width] [-l length] [-f file.root] [-C min_wav]
       [-d energy] [-a coverage]

Options:

-h:     This menu.
-j:     Wavelet scale, a number in the range 0 <= j <= 8,
        defaults to 1.
-N:     Band limit of wavelet, in the range 0 < N <= 128,
        defaults to 1.
-n:     Number of events in the simulated sky, defaults
        to n = 1000
-s:     Number of skies to simulate, defaults to 100.
-i:     Minimum energy of events, defaults to 20 EeV.
-e:     Maximum energy of events, defaults to 40 EeV.
-c:     Minimum correlation, defaults to 0.2.
-C:     Minimum Magniftude of wavelet coefficient,
        defaults to 0.0.
-m:     Minimum number of events hitting tangent plane.
-w:     Width of tangent plane, defaults to 2 degrees.
-l:     Length of tangent plane, defaults to 10 degrees.
-f:     Add events in TTree stored in file to the
        simulated sky.
-d:     Histogram with energy distributions.
-a:     Histogram with theta and phi distributions.

\end{lstlisting}
}

\subsection{swat\_gen}
Generates distribution of $E$ $\theta$ and $\phi$ coordinates that
will be used by swat\_sim program. Two kinds of distribution are
supported, isotropic or Auger distribution if herald data is provided.
\begin{figure}[ht]
   \centering
      \includegraphics[scale=0.25]{energy.jpg}
      \includegraphics[scale=0.25]{phi_theta.jpg}
   \caption{Energy distribution on the left. On the right
   angular distribution. Both for Auger experiment.}
   \label{healpix}
\end{figure}


\subsection{swat\_prob}
{\bf \color{textcolor}
   \begin{lstlisting}
   Reads a file output by swat_sim and calculates
   probabilities.  Use the macros cprob.tex and
   wprob.tex in pgfplots directory to produce
   probability graphs.

   Usage: swat_prob [-n n_steps] [-o outpu-file] [-t] 
                    [-k kind]

   Options:

      -h:     This menu.
      -o:     Output file, where probabilities
              will be recorded.
      -t:     Calculates P(q >= q_0) if provided,
              P(q < q_0) otherwise.  q will be the
              correlation or wavelet coefficient,
              depending on the value passed in option -k
      -k:     Either 1 for correlation or two for
              wavelet coefficient.
      -n:     Total number of steps (points on
              probability graph). Defaults to 10.

   \end{lstlisting}
}

\subsection{swat}
This program is only used to benchmark and test the algorithm. It can test
both the spherical harmonic transform and the spherical wavelet transform.
{\bf \color{textcolor}
\begin{lstlisting}
$ time swat -J 8 # Will test spherical harmonic transform.
$ time swat -J 8 -N 3 # Will test wavelet transform.
\end{lstlisting}
}

Help message
{\bf \color{textcolor}
   \begin{lstlisting}
   Tests the algorithm performing forward and backward
   transform.  Both spherical harmonic transform (if option
   -N is not provided) or spherical wavelet transforms can
   be performed. I use this program to benchmark my code
   using the time command:
   
   $ time swat -J 8 -N 127
   
   for example. If forward folloed by backward transform do
   not result in the data, with precision 1e-10, program
   exits with EXIT_FAILURE status. For example
   
   $ swat J8 -N 127
   $ echo $?
   0
   $
   
   Usage: swat [-J j] [-N n]

   Options: 
   -h:     This menu.
   -J:     Sets band limit of the signal to 2^J, defaults to
           J = 7.
   -N:     Band limit of wavelet to be used.

   \end{lstlisting}
}

\begin{thebibliography}{99}
\bibitem{frm} J Fourier Anal Appl (2008) 14: 145–179.
\bibitem{wiaux} Mon. Not. R. Astron. Soc. 000, 1–22 (2007). 
\bibitem{swat} \url{http://www.ifi.unicamp.br/~mzimbres/}
\end{thebibliography}
\end{document}

