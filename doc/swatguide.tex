\documentclass[12pt]{report}
\usepackage{tikz}
\usepackage{verbatim}
\usepackage{listings}
   
\lstset{ %
language=C++,                % the language of the code
basicstyle=\footnotesize,       % the size of the fonts that are used for the code
keywordstyle=\color{black}\bfseries,
%numbers=left,                   % where to put the line-numbers
%numberstyle=\footnotesize,      % the size of the fonts that are used for the line-numbers
%stepnumber=2,                   % the step between two line-numbers. If it's 1, each line 
%                                % will be numbered
%numbersep=5pt,                  % how far the line-numbers are from the code
backgroundcolor=\color{gray!20!white},  % choose the background color. You must add \usepackage{color}
%showspaces=false,               % show spaces adding particular underscores
showstringspaces=false,         % underline spaces within strings
showtabs=false,                 % show tabs within strings adding particular underscores
frame=single,                   % adds a frame around the code
%tabsize=2,                      % sets default tabsize to 2 spaces
%captionpos=b,                   % sets the caption-position to bottom
%breaklines=true,                % sets automatic line breaking
%breakatwhitespace=false,        % sets if automatic breaks should only happen at whitespace
title=\lstname,                 % show the filename of files included with \lstinputlisting;
% also try caption instead of title
%escapeinside={\%*}{*)},         % if you want to add a comment within your code
%morekeywords={*,...}            % if you want to add more keywords to the set
}

\usetikzlibrary{arrows}
\usetikzlibrary{shapes.multipart}
\usetikzlibrary{%
   decorations.fractals%
  ,decorations.pathmorphing%
  ,shadows%
}

\usepackage{hyperref}
\hypersetup{
bookmarks=false,         % show bookmarks bar?
unicode=false,          % non-Latin characters in Acrobat's bookmarks
pdftoolbar=true,        % show Acrobat's toolbar?
pdfmenubar=true,        % show Acrobat's menu?
pdffitwindow=true,     % window fit to page when opened
pdfstartview={FitH},    % fits the width of the page to the window
pdftitle={O Framework ROOT},    % title
pdfauthor={Marcelo Zimbres Silva},     % author
pdfsubject={Encontro C++ Brasil},   % subject of the document
pdfcreator={Marcelo Zimbres Silva},   % creator of the document
pdfproducer={Marcelo Zimbres Silva}, % producer of the document
pdfkeywords={C++} {ROOT}, % list of keywords
pdfnewwindow=true,      % links in new window
colorlinks=true,        % false: boxed links; true: colored links
linkcolor=red,          % color of internal links
citecolor=green,        % color of links to bibliography
filecolor=magenta,      % color of file links
urlcolor=cyan!50!black           % color of external links
}

% My definitions
%______________________________________________________________________

%\colorlet{swat}{yellow!50!black!50}
\colorlet{swat}{brown!50!black!50} % box color
\colorlet{textcolor}{brown!20!black} % text in box color
\colorlet{swatred}{red!50!black}

\tikzstyle{one}=[shade,top color=swat,fill=swat,rounded corners, textcolor,thick]
\tikzstyle{onel}=[rotate=-10,shade,top color=swat,fill=swat,rounded corners, textcolor,thick]
\tikzstyle{oner}=[rotate=10,shade,top color=swat,fill=swat,rounded corners, textcolor,thick]
\tikzstyle{inherit}=[open triangle 45-,thick,textcolor]
\tikzstyle{depend}=[draw,->,thick,textcolor,dashed]
\tikzstyle{code}=[draw,shade,rounded corners,text width=9cm]


\tikzstyle{two}=[anchor=north,shade,top color=swat,rectangle split, 
                     rectangle split parts=2, draw, text width=2.0cm,
		     fill=swat,rounded corners,textcolor,thick]

\tikzstyle{twonowidth}=[anchor=north,shade,top color=swat,rectangle split, 
                     rectangle split parts=2, draw, fill=swat,rounded corners,
		     textcolor,thick]


\tikzstyle{twored}=[anchor=north,shade,top color=swatred,rectangle split,
                        rectangle split parts=2,draw,text width=2.0cm,fill=swatred]

\tikzstyle{three}=[anchor=north,shade,top color=swat,rectangle split, rectangle split parts=3, draw, text width=3.0cm,fill=swat,
rounded corners,textcolor,thick]

\def\one{\node[style=one]}
\def\onel{\node[style=onel]}
\def\oner{\node[style=oner]}
\def\two{\node[style=two]}
\def\twonowidth{\node[style=twonowidth]}
\def\code{\node[style=code]}

\def\inherit{\draw[style=inherit]}
\def\depend{\draw[style=depend]}

\def\twored{\node[style=twored]}
\def\three{\node[style=three]}

%______________________________________________________________________
\begin{document}

  \parindent0pt
  \null
  \colorlet{mintgreen}{green!50!black!50}

  \thispagestyle{empty}
  \vskip3cm
  \vfill
  \hfil
  \begin{tikzpicture}[overlay]
    \coordinate (front) at (0,0);
    \coordinate (horizon) at (0,.28\paperheight);
    \coordinate (bottom) at (0,-.6\paperheight);
    \coordinate (sky) at (0,.57\paperheight);
    \coordinate (left) at (-.51\paperwidth,0);
    \coordinate (right) at (.51\paperwidth,0);

    \shade [bottom color=blue!30!black!10,top color=blue!30!black!50] ([yshift=-5mm]horizon -|  left) rectangle (sky -| right);
    \shade [bottom color=black!70!green!25,top color=black!70!green!10] (front -| left) -- (horizon -| left)
      decorate [decoration=random steps] { -- (horizon -| right) } -- (front -| right) -- cycle;
    \shade [top color=black!70!green!25,bottom color=black!25] (front -| left) -- (horizon -| left)
      decorate [decoration=random steps] { -- (horizon -| right) } -- (front -| right) -- cycle;
    \shade [top color=black!70!green!25,bottom color=black!25] ([yshift=-0mm]front -| left) rectangle ([yshift=1pt]front -| right);
    \fill [black!25] (bottom -| left) rectangle ([yshift=1mm]front -| right);

    \def\nodeshadowed[#1]#2;{\node[scale=2,above,#1]{#2};\node[scale=2,
        above,#1,yscale=-1,scope fading=south,opacity=0.4]{#2};}

    \node[at={(-0,6  )}] {\Huge \textsc{\textcolor{red}{SWAT}}}; 
    \node[at={(-0,4  )}] {\Huge {The \textcolor{red}{S}pherical \textcolor{red}{W}avelet \textcolor{red}{A}nalysis \textcolor{red}{T}ool}};
    \node[at={( 0,2  )}] {\Large Manual for version 2.32};

    \one (TObject) at (-8,15) {\Large TObject};
    \one (TArrayD) at (-3,15) {\Large TArrayD};
    \one (TDKernel) at (2,15) {\Large TDKernel};
    \one (TCoeffInfo) at (7,15) {\Large TCoeffInfo};

    \one (TNamed) at (-8,13) {\Large TNamed};
    \one (TVMap) at (-3,13) {\Large TVMap};
    \one (TEulerAngle) at (2,13) {\Large TEulerAngle};
    \one (TAngDistance) at (7,13) {\Large TAngDistance};

    \one (TSkyMap) at (-8,11) {\Large TSkyMap};
    \one (TWignerd) at (-3,11) {\Large TWignerd};
    \one (TWavMap) at (2,11) {\Large TWavMap};
    \one (TAlm) at (7,11) {\Large TAlm};

    \one (TSphHarmB) at (-8,9) {\Large TSphHarmB};
    \one (TSwatB) at (-3,9) {\Large TSwatB};
    \one (TSwatF) at (2,9) {\Large TSwatF};
    \one (TSphHarmF) at (7,9) {\Large TSphHarmF};
   
   %_______________________________________________________
   \draw[draw,-open triangle 45,thick,color=gray] (node cs:name=TNamed) -- (node cs:name=TObject);
   \draw[open triangle 45-,thick,gray] (node cs:name=TDKernel,anchor=south) |- +(0,-1)  node (y) {};
   \draw[open triangle 45-,thick,gray] (node cs:name=TCoeffInfo,anchor=south) |- +(0,-1)-- +(-10.0,-1);
   \draw[open triangle 45-,thick,gray] (node cs:name=TArrayD,anchor=south) -- (node cs:name=TVMap);
   \draw[draw,-open triangle 45,thick,gray] (node cs:name=TVMap) -- (node cs:name=TNamed);
   \draw[draw,-open triangle 45,thick,gray] (node cs:name=TSkyMap) -- (node cs:name=TVMap);
   \draw[draw,-open triangle 45,thick,gray] (node cs:name=TWavMap) -- (node cs:name=TVMap);
   \draw[draw,->,thick,gray,dashed] (node cs:name=TWavMap) -- (node cs:name=TSwatB) node[midway,sloped,above]           {};
   \draw[draw,->,thick,gray,dashed] (node cs:name=TWavMap) -- (node cs:name=TAlm) node[midway,sloped,above]             {};
   \draw[draw,->,thick,gray,dashed] (node cs:name=TSkyMap) -- (node cs:name=TSphHarmB) node[midway,sloped,above]        {};
   \draw[draw,->,thick,gray,dashed] (node cs:name=TSwatB) -- (node cs:name=TWignerd) node[midway,sloped,above]          {};
   \draw[draw,->,thick,gray,dashed] (node cs:name=TSphHarmB) -- (node cs:name=TWignerd) node[midway,sloped,above]       {};
   \draw[draw,->,thick,gray,dashed] (node cs:name=TWavMap) -- (node cs:name=TEulerAngle) node[midway,sloped,above]      {};
   \draw[draw,->,thick,gray,dashed] (node cs:name=TAngDistance) -- (node cs:name=TEulerAngle) node[midway,sloped,above] {};
   \draw[draw,->,thick,gray,dashed] (node cs:name=TAlm) -- (node cs:name=TSwatF) node[midway,sloped,above]              {};
   \draw[draw,->,thick,gray,dashed] (node cs:name=TAlm) -- (node cs:name=TSphHarmF) node[midway,sloped,above]           {};

   %_______________________________________________________

    \node[text width=15cm] at (-1,-5.5) 
    { \bf\small
    Double\_t TWignerd::Delta(Int\_t m,Int\_t n) const \newline
    \{ \newline 
    \textcolor{white}{\ \ \ if} (m $<$ 0) \{\newline
          \textcolor{white}{\ \ \ \ \ \ if} (n $<$ 0) \{\newline
             \textcolor{white}{\ \ \ \ \ \ \ \ \ if} (m $>=$ n)\newline
                \textcolor{white}{\ \ \ \ \ \ \ \ \ \ \ \ return} fMatrix[fIndex(-n,-m)];\newline
             \textcolor{white}{\ \ \ \ \ \ \ \ \ return} Even(n-m) ? fMatrix[fIndex(-m,-n)]: -fMatrix[fIndex(-m,-n)];\newline
        \textcolor{white}{\ \ \ \ \ \ }  \} else \{\newline
             \textcolor{white}{\ \ \ \ \ \ \ \ \ if} (n $<=$ Abs(m))\newline
                \textcolor{white}{\ \ \ \ \ \ \ \ \ \ \ \ return} Even(fL-n) ? fMatrix[fIndex(-m,n)]: -fMatrix[fIndex(-m,n)];\newline
    	 \textcolor{white}{\ \ \ \ \ \ \ \ \ return} Even(fL-m) ? fMatrix[fIndex(n,-m)]: -fMatrix[fIndex(n,-m)];\newline
        \textcolor{white}{\ \ \ \ \ \ }  \}\newline
       \textcolor{white}{\ \ \ }\}\newline
       \textcolor{white}{\ \ \ \ \ \ if} (n $<$ 0) \{\newline
          \textcolor{white}{\ \ \ \ \ \ \ \ \ if} (m $>=$ Abs(n))\newline
             \textcolor{white}{\ \ \ \ \ \ \ \ \ \ \ \ return} Even(fL-m) ? fMatrix[fIndex(m,-n)]: -fMatrix[fIndex(m,-n)];\newline
          \textcolor{white}{\ \ \ \ \ \ \ \ \ return} Even(fL-n) ? fMatrix[fIndex(-n,m)]: -fMatrix[fIndex(-n,m)];\newline
       \textcolor{white}{\ \ \ \ \ \ }\} else \{\newline
          \textcolor{white}{\ \ \ \ \ \ \ \ \ if} (m $>=$ n)\newline
             \textcolor{white}{\ \ \ \ \ \ \ \ \ \ \ \ return} fMatrix[fIndex(m,n)];\newline
          \textcolor{white}{\ \ \ \ \ \ \ \ \ return} Even(n-m) ? fMatrix[fIndex(n,m)]: -fMatrix[fIndex(n,m)];\newline
     \textcolor{white}{\ \ \ }  \}\newline
    \}\newline

    };
  \end{tikzpicture}
  \vfill


\newpage
\tableofcontents

\chapter{Introduction} \label{ch::Introduction}

The software \href{http://www.ifi.unicamp.br/~mzimbres}{SWAT} is a package 
for analysis of function that live on the sphere. I have written it from
scratch as part of my my Ph.D. project. It evolved from a simple ROOT macro and
got bigger and bigger, since the code may be useful for other people I
decided to organize, document it and make it public. The code is an C++ implementation of
the spherical wavelet transform as presented in \cite{wiaux}. Some attractive
features of SWAT are:

\begin{itemize}
\item Harmonic and Wavelet transform on the sphere.
\item Easy calculation of Healpix maps form Herald data and CRPropa simulations.
\item Easy selection of events hitting sky window.
\item Calculation of deflection vs. $1/E$ graphs.
\item Calculation of the Wigner-d functions via FFT.
\item ROOT is the only prerequisite (with module FFTW enabled).
\item You can load all the code in a ROOT session and call SWAT code in your ROOT macro.
\item You can save SWAT objects to .root files.
\end{itemize}

The package includes some parts of the Healpix code. There are two reasons
why I decided to include it here, instead of just link against Healpix
libraries.
\begin{enumerate}
\item I do not need all Healpix routines and support to the fits format. 
\item I usually need shared libraries the call Healpix code in a ROOT session, which is not built by
Healpix build system since most of its code is C++ templates.
\end{enumerate}

Most of the theoretical datails of analysis of functions defined on $S^2$ and $S^3$ 
were taken from \cite{wiaux,rockmore,navaza}. 

The code has been tested in \textsc{Linux} and \textsc{MacBook}, but the code
is portable enough to be built on other platforms.  In the following we provide
some theoretical foundations which should be enough to the physicist wanting to
understand the math behind the implementation, for more thoroughly description
refer to the bibliography. Questions concerning the software can sent to
Marcelo Zimbres \href{mailto:mzimbres@gmail.com}{mzimbres@gmail.com}
\vspace{1cm}
\newline
\subsubsection{Acknowledgements}
SWAT is being developed with financial support from CAPES.  Additionaly I would
like to acknowledge the Institute of Physics at UNICAMP for finacial support,
Brunel University for granting me scholarship to participate in the Cern School
of Computing, the Ettore Majorana Foundation and Center for scientific Culture, the brasilian
group of the Pierre Auger experiment, with special thanks to my supervisor
Ernesto Kemp, for beliving in the project and Rafael Alves Batista for testing
the code. The Pierre Auger Group at university of Wuppertal, with special
thanks to Nils Niertenh\"ofer for helping me with CRPropa.

\section{Installation} \label{ch::installation}
As a prerequisite to install SWAT you need ROOT installed with FFTW module
enabled, the configure script will complain if they are not installed. SWAT
uses autotools to generate the configure and Makefile files, so you can
expect all the standard configure options and makefile targets. The lines 
bellow will install the libraries on {\color{brown}"/usr/local"}
{ \color{brown}
   \begin{lstlisting}
   $ ./configure
   $ make
   $ make install
   \end{lstlisting}
}

To use some of the features of the package, I am assuming you are able to
build shared libraries on your platform.  To easy the task of loading
swat libraries the macro load.C, macro will be installed on {\color{brown}"prefix/share/swat"}. To
load the libraries in ROOT's C++ interpreter you just have to copy this macro
to your working directory, or just add it to your code {\color{brown}.rootlogon.C} macro.

{ \color{brown}
\begin{lstlisting}
$ cp /usr/local/shar/swat/load.C ~/.rootlogon.C
\end{lstlisting}
}

All SWAT classes are now available in the Root session with syntax
highlighting. You should also be able to generate documentation in html format
with the macro {\color{brown}"prefix/share/swat/makehtml.C"}. You have to run
this macro from the directory where you built SWAT. The documentation will be
built in the directory{\color{brown}htmldoc}.

\section{Getting ready to run the analysis} \label{ch::ready}
To analise Herald data, you will have to convert the ascii file to a .root
file, for that you should use the macro
{\color{brown}prefix/share/swat/convert\_herald.C} where prefix is usually
\textit{/usr/local}.  Copy this macro to the same directory where you have the
herald data file. The macro will read a file with name
{\color{brown}"herald.dat"}, so you may have to rename it. 

For CRPropa simulations you can just configure CRPropa to output a .root file
instaed of a text file.  The code uses the title of the TTree to diferentiate
between a Herald and CRPropa file. The title of the TTree for CRPropa must be
\textit{\color{brown}"CRPropa 3D events"}.

\section{High Quality graphs with pgfplots} \label{ch::pgfplots}
If you have pgfplots installed on your you can use the \LaTeX files installed in 
{\color{brown}"prefix/share/swat"} to generate high quality graphs. You may have noted that 
after running {\color{brown}TAuxFunc::find\_sources()} and {\color{brown}TAuxFunc::find\_multiplets()}
you get many text files in your working directory. They are input files for the \LaTeX files. Here are some 
example graphs \\
\vspace{1cm}
\includegraphics[scale=0.5]{fig/skymap-herald-f1.pdf} \hfill
\includegraphics[scale=0.5]{fig/corr_graph_crpropa-f5.pdf} \hfill
\includegraphics[scale=0.5]{fig/multiplicity-crpropa-f2.pdf} \\ 

\chapter{Analysis algorithms}
\section{Finding sources}

A source here is a term used to mean something with a position on the sky, usualy described
by $(\theta,\phi)$ and an orientation. We will use the three Euler angle $(\alpha,\beta,\gamma)$ to
describe the source. You can use the following code to find sources in your data.
{ \color{brown}
   \begin{lstlisting}
   //   Parameter description.
   //
   //   emin: char string with minimum energy, e. g. "15"
   //   emax: char string with maximum energy, e. g. "40"
   //   N: Wavelet band limit.                           
   //   j: Wavelet scale                                 
   //   nsource: Number of sources to look for.          
   //   w: Source minimum separation in degrees.         
   //   file: Herald file containing the TTree.          
   $ root
   root [1] TAuxFunc::find_sources(emin,emax,N,j,nsources,w,file)
   \end{lstlisting}
}
For example:
{ \color{brown}
   \begin{lstlisting}
   root [1] find_sources("15","40",111,2,10,5,"herald.root")
   \end{lstlisting}
}

The algorithm goes as follows:
\begin{enumerate}
\item The TTree is read selecting events with energies in the range $emin < E < emax$.
\item A Healpix map is generated and transformed to wavelet space. The parameter $N$
gives the precision on the ability of the wavelet to find the angle $\gamma$ and $j$
is the scale at witch the wavelet transform is performed. Refer to \cite{gap} for
the meaning of these parameters.
\item A partial sort is used to find the nsources first biggest wavelet coefficients.
The position of the wavelet coefficient, described by the three euler angles is
saved in a TEulerAngle object and saved in the file {\color{brown}"sources.root"}
\end{enumerate}
Once you have a file with the sources objectss, you can see the source position using
the method {\color{brown}TEulerAngle::Show}
{ \color{brown}
\begin{lstlisting}
$ root sources.root
root[1] gDirectory->ls()
TFile**		sources.root	
TFile*		sources.root	
KEY: THealpixMap	hmap;1	Healpix sky map
KEY: TEulerAngle	source0;1	
KEY: TEulerAngle	source1;1	
KEY: TEulerAngle	source2;1	
root [2] source0->Show(1)
wav(137.812,-27.4219,159.638) = -0.035312
root [3] 
\end{lstlisting}
}

\subsection{Implementation}

\section{Deflection vs. Energy} \label{ch::raios-cosmicos}
You can also easily calculate graphs of deflection versus $1/E$ this way:
{ \color{brown}
   \begin{lstlisting}
   root[1] TAuxFunc::find_multiplets(l,w,"sources.root","chain.root")
   \end{lstlisting}
}
This code will:
\begin{enumerate}
\item Read the sources file {\color{brown}"sources.root"}
\item For each source it will calculate the equations descibing a stripe of
length $l$, width $w$ tangent to the sources and aligned with it.
\item Read the Herald data, select all events hitting the stripe and calculate 
the deflection versus $1/E$ graphs.
\item The graphs will be added to the sources file.
\end{enumerate}


{ \color{brown}
   \begin{lstlisting}
   $ root sources.root
   root[1] gDirectory->ls()
   TFile**		sources.root	
   TFile*		sources.root	
   KEY: THealpixMap	hmap;1	Healpix sky map
   KEY: TEulerAngle	source0;1	
   KEY: TEulerAngle	source1;1	
   KEY: TEulerAngle	source2;1	
   KEY: TEulerAngle	source3;1	
   KEY: TGraphErrors	g0;1	C = 0.216, N = 22
   KEY: TGraphErrors	g1;1	C = 0.294, N = 15
   KEY: TGraphErrors	g2;1	C = -0.013, N = 23
   KEY: TGraphErrors	g3;1	C = -0.005, N = 42
   root [2] g0->GetCorrelation()
   root [2] g0->Fit("pol1")
   root [2] g0->Draw("ap")
   \end{lstlisting}
}

The graph with name $g0$ correspond to source $source0$ and so on.
The function of the last section will also calculate the "Number of events vs. stripe orientation"
and add to the sources.root file. The graphs will be named $g0$, $g1$ ...

\subsection{Implementation}
\section{Counting in stripes} \label{ch::raios-cosmicos}


\chapter{More code}
\section{Wigner-d functions}
\section{Drawing Spherical harmonics}
\section{Drawing wavelets}

\begin{thebibliography}{99}
\bibitem{frm} J Fourier Anal Appl (2008) 14: 145–179.
\bibitem{wiaux} Mon. Not. R. Astron. Soc. 000, 1–22 (2007). 
\bibitem{swat} \url{http://www.ifi.unicamp.br/~mzimbres/}
\end{thebibliography}
\end{document}



